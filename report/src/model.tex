\section{Model and strategy} % (fold)
\label{sec:model_and_strategy}

\subsection{Board description} % (fold)
\label{sub:board_description}

% \todo{This subsection maybe does not need to stay here.}
% 
% \paragraph{Change of square values and traps definition} % (fold)
% \label{par:change_of_square_values_and_traps_definition}
% 
% % paragraph change_of_square_values_and_traps_definition (end)
% 
% \paragraph{No consecutive traps} % (fold)
% \label{par:no_consecutive_traps}
% A first assumption in the way we defined our board concerns the non-existence
% of domino effect of traps. This means that the only way to activate a trap
% is to fall on its box immediately after one dice throw.
% For example, if one lands on box 8, which is a type 3 [??? type 2 if we can not
% modify the project rules] trap, it brings us back to box 5. In this case,
% even if box 5 is a trap, it won't affect us until the following dice throw
% (i.e. if a 0 is made at the following dice throw).
% % paragraph no_consecutive_traps (end)
% 
% % subsection board_description (end)

\subsection{Markov model} % (fold)
\label{sub:markov_model}
Let us first recall the general idea behind Markov decision processes.
The objective is to determine the best set of actions $\{u(1), u(2), \dots
u(n)\}$ that minimizes $V_{\pi}(s_0)$, the total expected cost of the process.
The algorithm we use in order to determine this set is the one based on Bellman equation
\[
  V(k) \leftarrow \min_{a\in U(k)} \{c(a|k) 
  + \sum_{k'=1}^n p(k'|k,a)V(k') \} \qquad k \neq d
\]
where $V(d)$ is set to 0 at each iteration, $d$ being the destination state.
In this value-iteration algorithm, $c(a|k)$ and $p(k'|k,a)$
respectively represent the \emph{cost} of making action $a$
when being in state $k$ and the \emph{probability} of going
from state $k$ to $k'$ when action $a$ is triggered.

Linking this process to our \textit{Snake and Ladder} game,
a state $k$ corresponds to \emph{being on} the kth square,
thus taking values in the range 1 to 15.
We then define the action $a$ as the \emph{throw of a dice}
and since every one of the three dices can be thrown at each
iteration we have $U(1) = \dots = U(15) = \{a_{\text{secu}},
a_{\text{norm}}, a_{\text{risk}}\}$.
Finally, recalling that the goal of the game is to get to
the 15th square by \emph{minimizing the number of dice throws},
the cost $c(a|k)$ equals 1 as each turn will lead in one dice
throw, whatever the action $a$ or the state $k$. 

At this point only one last ingredient is missing to start implementing
the above described algorithm : the transition probabilities $p(k'|k,a)$.
Here lies the first main difficulty of the project as those depends
both on the type of square or trap from which one leaves $k$ and
on which one will arrive $k'$, as well as on the dice one
decides to use (i.e. $a$).
It is now important to \emph{find an efficient way of constructing
the 3 probability matrices} $P_{\text{secu}}$, $P_{\text{norm}}$
and $P_{\text{risk}}$ by minimizing the number of particular cases,
as one can not even imagine to compute the $3\times 15^2$ individual probabilities.

The precise implementation of the probability matrices won't be
discussed in this section but it appeared important to us
to still make a point on the mathematical correctness of our method.
The idea of our method can be decomposed in two steps.
\begin{enumerate}
  \item First fill the matrix by assuming that
  all squares are normal squares and thus filling the matrices
  with probabilities $\frac{1}{2}$, $\frac{1}{3}$ and $\frac{1}{4}$,
  making particular cases for the slow/fast \emph{lanes}
  and the \emph{circularity} of the board.
  \item Then re-modify each matrix line by taking traps into account
  and thus re-distributing the traps transition probabilities.
  For example, if there a \emph{type 2 trap} at square 8 
  and the \emph{normal} dice is thrown, we make the following adjustments
  for each line $i$.
  \begin{align*}
    p(i,5) &\leftarrow p(i,5) + p(i,8)/2 \\
     p(i,8) &\leftarrow p(i,8)/2
  \end{align*}
\end{enumerate}

% subsection markov_model (end)

\subsection{Choosing the optimal dice} % (fold)
\label{sub:choosing_the_optimal_dice}

% subsection choosing_the_optimal_dice (end)

% section model_and_strategy (end)