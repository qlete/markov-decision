\section{Implementation} % (fold)
\label{sec:implementation}
Now that the model and strategy has been described in the previous
section, we will focus on the \emph{implementation}.
As mentioned in the introduction, we decided to use the \textsc{Julia}
programming language as our coding partner.
Using \textsc{Julia} intuitive package generator, we were able to
produce beautiful documentation and we therefore invite
the reader to have a look at the \textsc{html} documentation located
in the folder \lstinline|html_doc/| by opening the \texttt{index.html} file.
The latter contains a structured presentation of the architecture
of our code as well as a description of each function.

We will now describe the global architecture of our solution. 
The reader interested in details about the implementation of
specific functions is referred to the \textsc{html} documentation. 

The first thing to do before being able to apply the algorithm
is to define the probability matrices for each dice. 
This is done in the file \texttt{dices.jl} that is decomposed into
one function per dice. 

Once this is done, we can apply the value-iteration algorithm itself. 
The implementation of the algorithm can be found in the file \texttt{markovDecision.jl}
which also contains the function mentionned in the project statement. 

Then, we have to implement the simulation of the game given a dice strategy.
To help in the simulation, we defined functions that simulate the effect
of each trap in the file \texttt{traps.jl}. The simulation itself is
implemented in the file \texttt{simulate.jl}. 

The file \texttt{perf\_plots.jl} contain functions that make the plots
of our report. 

Finally, the file \texttt{SnakeLadder.jl} defines the \textsc{Julia} module
necesssary to generate a proper package. 

There is also an additional file called \texttt{experiments.jl}
which launches some simulations of the game and prints the results.
The user interested in testing our implementation and get quick results
is invited to execute this file.

% section implementation (end)