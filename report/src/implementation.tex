\section{Implementation} % (fold)
\label{sec:implementation}
Now that the model and strategy have been described in the previous
section, we will focus on the \emph{implementation}.
We decided to use the \textsc{Julia}
programming language as our coding partner.
Using \textsc{Julia} intuitive package generator, we were able to
produce beautiful documentation and we therefore invite
the reader to have a look at the \textsc{html} documentation located
in the folder \lstinline|html_doc/| by opening the \texttt{index.html} file.
The latter contains a structured presentation of the architecture
of our code as well as a description of each function.

We will still here describe the global architecture of our solution,
recalling that the reader interested in details about the implementation of
specific functions is referred to the \textsc{html} documentation. 

\begin{itemize}
  \item The first thing to do before being able to apply the algorithm
  is to define the \emph{probability matrices} for each dice. 
  This is done in~\texttt{dices.jl} that is decomposed into one function per dice. 
  
  \item We then apply the \emph{value-iteration algorithm} itself and its
  implementation can be found in the file~\texttt{markovDecision.jl}
  which also contains the required function of the project statement. 
  
  \item Then, we have to implement the \emph{simulation} of the game given a dice strategy.
  To help in the simulation, we defined functions that simulate the effect
  of each trap in the file \texttt{traps.jl}. The simulation itself is
  implemented in the file \texttt{simulate.jl}. 
  
  \item Next, the file~\texttt{perf\_plots.jl} contains functions
  that make the plots used in our report. 
  
  \item An additional file, \texttt{experiments.jl}, launches some simulations
  on various boards and prints the results in a clear fashion.
  Executing this file will allow to quickly verify our implementation.
  
  \item Finally, \texttt{SnakeLadder.jl} defines the \textsc{Julia} module
  necessary to generate a proper package. 
\end{itemize}

% section implementation (end)